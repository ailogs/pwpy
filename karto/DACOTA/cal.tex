\documentclass[preprint]{aastex}
\usepackage{amsmath, amsfonts, amssymb}

\begin{document}

\title{Calibration of the DACOTA Ku-BAND Prototype System}
\author{Garrett "Karto" Keating \\ \today}

\begin{abstract} Calibration is tricky. Calibration of a super awesome telescope, doubly so. Here we explore some of the parameters to be explored with, well, you know.
\end{abstract}

\section{DACOTA System Overview}\label{secoverview}
Let us do some informal calculations regarding the proposed DACOTA system design. From current design specifications, a single DACOTA element for the prototype will consist of a 1.8 m primary, with a single polarization Ku-band feed (15-18 GHz). The single element is expect to have an overall efficient of $\eta=0.6$, and a system temperature of 100 K.

Combining these numbers, we find that a single element has a system equivalent flux density given by the following equation.
\begin{equation} \label{eqsefd}
S_{sys}=\frac{1}{2}G_{ant}{\eta}T_{sys}
\end{equation}

\noindent A factor of one-half is introduced due to the fact that we only detect a single polarization. $G_{ant}$ is the gain of the antenna, which for a 1.8 meter primary with $\eta=0.6$ is 1809 Jy/K. This gives an SEFD of $S_{ant}=181\text{ kJy}$. Assuming a minute of integration time\footnote{An estimate of one minute for calibration is made to conservatively limit the number of calibrators considered adequate for calibration. Obviously, more time for calibration can be allocated, which will in turn increase the number of viable calibrators (although not likely by a significant amount).}, for a single baseline we expect to see a signal to noise ratio of $SNR=\frac{S_{Cal}}{0.738\text{ Jy}}$. Assuming an array of 19 elements packed into a hex configuration\footnote{Giving a synthesized beamwidth of approximately 8.5 arcminutes}, integrating over a minute with a bandwidth of 1 GHz will produce an image with a theoretical RMS noise of 39.9 mJy/beam. 

Our primary science driver - the detection of CO within star-forming regions of primordial galaxies - requires that we integrate down to a level of 1 microKelvin in a channel of velocity width of 100 km/s. Assuming 264 elements in the full array, with a system temperature of 50 K for all antennas, this means that 420 minutes of integration time will be required per pointing, which will produce an image with a theoretical noise RMS of 0.967 mJy/beam. A one minute integration on a calibrator will produce an image with a theoretical noise RMS of 19.8 mJy/beam.

Assuming that the primary contribution to the system temperature is in the preamp, then we expect 


\end{document}

