\documentclass[preprint]{aastex}
\usepackage{amsmath, amsfonts, amssymb, url, natbib}

\begin{document}

\title{Calibration of the DACOTA Ku-BAND Prototype System}
\author{Garrett "Karto" Keating \\ \today}

\begin{abstract} In this memo, we explore some of the challenges and requirements for calibration of the 19-element prototype of the Discovery Array for Cosmology and Transient Astrophysics (DACoTA). Overall, we find that there are 10 suitable astronomical calibrators for wideband phase and amplitude solutions, and 3 suitable astronomical calibrators for bandpass solutions. The distribution of these calibrators across the sky ensures that there will be at least one of each calibrator type available for any LST. Additionally, we explore the potential exploitation of Ku-band transmitters for the purposes of phase and primary beam calibration.
\end{abstract}

\section{DACOTA System Overview}\label{secoverview}
Let us do some informal calculations regarding the proposed DACOTA system design. From current design specifications, a single DACOTA element for the prototype will consist of a 1.8 m primary, with a single polarization Ku-band feed (15-18 GHz). The single element is expect to have an overall efficiency of $\eta=0.6$, and a system temperature of 100 K. Combining these numbers, we find that a single element has a system equivalent flux density given by equation \ref{eqsefd}.
\begin{equation} \label{eqsefd}
S_{sys}=\frac{1}{2}G_{ant}{\eta}T_{sys}
\end{equation}

\noindent A factor of one-half is introduced due to the fact that we only detect a single polarization in our feed. $G_{ant}$ is the gain of the antenna, which for a 1.8 meter primary with $\eta=0.6$ is 1809 Jy/K. This gives an SEFD of $S_{ant}=181\text{ kJy}$. Assuming two minutes of integration time\footnote{An estimate of two minutes for calibration is made to conservatively limit the number of calibrators considered adequate for calibration. Assuming calibration is needed twice an hour, and taking into account slewing time and time to initiate datataking, this means that approximately 10\% of observing time will be spent in calibration.} with 1 GHz of bandwidth, for a single baseline we expect to see a signal to noise ratio of $R_{NS}=\frac{S_{Cal}}{0.739\text{ Jy}}$. Assuming an array of 19 elements packed into a hex configuration, integrating over two minutes with a bandwidth of 1 GHz will produce an image with a theoretical RMS noise of 39.9 mJy/beam, with a resolution of 6.87 arcminutes. An individual channel of 16 MHz under the same constraints will produce an image with a theoretical RMS noise of 316 mJy/beam. 

Our primary science driver - the detection of CO within star-forming regions of primordial galaxies - requires that we integrate down to a level of 1 microKelvin in a channel of velocity width of 100 km/s. The prototype will have a velocity resolution of roughly 300 km/s. Assuming that the brightness temperature constraint is the same for both velocity resolutions, a single pointing will require 1767 hours of observing time, and will produce an image with a theoretical noise RMS of 0.967 mJy/beam. For comparison, the full 264 element array with a system temperature of 50 K for all elements will require 7 hours of integration per pointing.

For a 1.8 meter primary, the DACOTA primary will have a field of view that covers roughly 0.258 square degrees, with sidelobes down to -30 dB (relative to the peak of the sythesized beam) extending over approximately 4.13 square degrees. We know the density of point sources as a function of flux density to be as described by equation \ref{eqsoudensity} \citep{MNR20109C}.
\begin{equation} \label{eqsoudensity}
n(S)=51 (S/\text{Jy})^{-2.15} \text{Jy}^{-1} \text{sr}^{-1}
\end{equation}
We expect to see a total of 135 mJy of flux from point sources above the noise threshold\footnote{This number represents the total flux over the sky weighted by the primary beam}, the brightest point source that we expect to see over the extended field of view is 280 mJy. This implies that DACOTA will need to achieve a minimum dynamic range of 140 in imaging during calibration, and preferably a dynamic range of at least 290.

Assuming that the primary contribution to the system temperature is in the preamp, then at 100 K with 3 GHz of bandwidth on the front-end of the system, we expect the power of the receiver noise to be on the order of -84 dBm. Though the system noise will significantly dominate over most celestial source for calibration, we will need to consider the nominal power moving through the front end for bright astronomical (e.g. the Sun) and man-made (e.g. DirecTV satellites) emitters.
\section{Astronomical Calibrators}\label{seccalibrate}

\subsection{Phase and Amplitude Calibration}\label{ssecphamp}
For the derivation of phase and amplitude gains solutions, we are restricted to looking at calibrators whose fluxes are above 5.6 Jy (although ideally 11.6 Jy) if we are limited to two minutes of calibration with the prototype. Presented in Table \ref{tablecal} are a list of viable calibrator choices for the prototype, the only constraint for which is that the calibrator exceed a minimum flux of 5.6 Jy. Overall, we find that there are a total of 10 calibrators suitable for this task.
\begin{table}[!h]
\begin{center}
\begin{tabular}{|c|c||c|c|c|c|c|c|c} \hline
 & & RA & Dec & 2-cm Flux& Rise & Set \\
Name & IAU Name & (J2000) & (J2000) & (Jy) & (LST) & (LST) \\
\hline
\hline
3C84 & 0319+415 & 03h19m48s & +41d30m42s & 20.7 & 18.34 & 12.32 \\
N/A & 0609-157 & 06h09m41s & -15d42m41s & 9.0 & 1.04 & 11.44 \\
N/A & 0927+390 & 09h27m03s & +39d02m21s & 6.60 & 0.81 & 18.10 \\
3C273 & 1229+020 & 12h29m07s & +02d03m09s & 34.0 & 6.37 & 18.60 \\
3C279 & 1256-057 & 12h56m11s & -05d47m22s & 21.8 & 7.26 & 18.62 \\
3C345 & 1642+398 & 16h42m59s & +39d48m37s & 13.0 & 7.95 & 1.48 \\
N/A & 1733-130 & 17h33m03s & -13d04m50s & 11.0 & 12.29 & 22.81 \\
N/A & 1924-292 & 19h24m51s & -29d14m30s & 17.0 & 15.17 & 23.66 \\
3C446 & 2225-049 & 22h25m47s & -04d57m01s & 6.60 & 16.69 & 4.17 \\
3C454.3 & 2253+161 & 22h53m58s & +16d08m54s & 15.0 & 15.98 & 5.82 \\
\hline
\end{tabular}
\caption{List of viable DACOTA Prototype calibrators. Values have been taken from the 2003 VLA calibrator database. Rise and set times are for Berkeley, California.
\label{tablecal}} 
\end{center}
\end{table}
From the list provided by the VLA Calibrator Database \citep{VLACalCat}, we find that there are at least three viable calibrators above the horizon at any given time. There are likely to be additional constrains on pointing (solar avoidance, terrain and pointing constraints, etc.), but this analysis suggests that there should be at least one viable calibrator choice at any given time for the DACOTA prototype.
\subsection{Bandpass Calibration}\label{ssecbandpass}
In addition to calculating wideband complex gains solutions, we will also need to calculate bandpass calibration solutions for the system. In order to meet our minimum imaging requirements, we will need to select calibrators which are bright enough to give us a dynamic range of 290 in an individual channel image. Fortunately, we do not expect the bandpass solutions to change drastically over the course of the day, so we will assume that we can integrate for twenty minutes instead of the nominal 2 minutes for calibration. This restricts us to looking at calibrators with fluxes of 29 Jy and above. A total of four viable bandpass 
\begin{table}[!h]
\begin{center}
\begin{tabular}{|c||c|c|c|c|c|c|c} \hline
 & RA & Dec & 2-cm Flux & Rise & Set & Size \\
Name & (J2000) & (J2000) & (Jy) & (LST) & (LST) & (arcmin) \\
\hline
\hline
Cassiopeia A & 23h23m24s & +58d48m54s & 307 & N/A & N/A & $4.3\times4.3$ \\
Cygnus A & 19h59m28s & +40d44m02s & 102 & 11.18 & 4.80 & $2.0\times2.0$ \\
Taurus A & 05h34m32s & +22d00m52s & 499 & 22.30 & 12.85 & $4.2\times2.6$ \\
Virgo A & 12h30m49s & +12d23m28s & 30 & 5.83 & 19.19 & $1.0\times1.5$ \\
\hline
\end{tabular}
\caption{List of viable DACOTA Prototype bandpass calibrators. Values taken from \cite{Baars65}. Rise and set times are for Berkeley, California.
\label{tablebandpass}} 
\end{center}
\end{table}
With the exception of a one hour block of time, there are at least two viable bandpass calibrators above the horizon at any given time. Cassiopeia A is a circumpolar target, and is small enough such that it should not be significantly resolved by the DACOTA prototype. As such, it would be an ideal primary choice for bandpass calibration.
\section{Unconventional Calibrator Choices}\label{secsats}
\subsection{Satellites as Phase and Amplitude Calibrators}\label{ssecphampsat}
\subsection{Satellites as Primary Beam Calibrators}\label{ssecpbcalsat}
%\section{Looking Forward: Additional Calibration Issues with DACOTA}\label{secfuture}
%\subsection{Polarization Calibration}\label{ssecpolcal}
%\subsection{Primary Beam Calibration}\label{ssecpbcal}
%\subsection{Increased Sensitivity and Calibration Models}\label{ssecmodels}
\section{Conclusion}\label{secconc}

\bibliographystyle{plainnat}
\bibliography{DACOTA}

\end{document}

